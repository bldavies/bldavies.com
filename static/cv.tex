\documentclass[11pt,oneside]{memoir}

\title{Curriculum Vitae}
\author{Benjamin Davies}
\date{\today}

\usepackage{mathpazo}
\usepackage[T1]{fontenc}
\usepackage[utf8]{inputenc}
\usepackage[margin=0.6in]{geometry}
\usepackage[colorlinks,urlcolor=blue]{hyperref}
\usepackage{titlesec}

\linespread{1.1}
\pagestyle{empty}
\raggedright
\setlength{\parindent}{0pt}
\setsecnumdepth{part}
\titleformat{\section}{\large\bfseries}{}{}{}[\titlerule]
\titleformat{\subsection}{\bfseries}{}{}{}
\titlespacing{\section}{0pt}{1.5em}{1em}
\titlespacing{\subsection}{0pt}{1em}{0pt}

\newcommand{\entry}[1]{\par\parbox[t]{\hsize}{\strut\raggedright #1}}
\newcommand{\datedentry}[2]{\par\parbox[t]{0.9\hsize}{\strut\raggedright\hangindent=2em #2\strut}\hfill#1}
\newcommand{\paper}[2]{\par\entry{#1 \\ \vskip0.25em \parbox[t]{\hsize}{\strut\small\emph{Abstract}: #2}}}

\begin{document}

{\Large\bfseries\MakeUppercase\theauthor} \hfill {\small\itshape\today}
\vskip0.25em
\hrule
\vskip1.5em

Department of Economics \hfill \href{mailto:bldavies@stanford.edu}{bldavies@stanford.edu} \\
Stanford University \hfill \href{https://bldavies.com}{bldavies.com} \\
Stanford, CA 94305, USA \hfill Citizenship: New Zealand

\section{Education}

\subsection{Stanford University}
\entry{PhD in Economics \hfill 2020--26 (Expected) \\
    {\small\emph{Fields}: Microeconomic Theory, Behavioral and Experimental Economics} \\
    {\small\emph{Committee}: Matthew Jackson (co-advisor), Arun Chandrasekhar (co-advisor), Steven Callander}
    }

\subsection{University of Canterbury}
\entry{BSc(Hons, 1st class) in Economics and Mathematics \hfill 2014--17}

\section{Job Market Paper}
\paper{\href{https://bldavies.com/jmp.pdf}{\bfseries The value of conceptual knowledge} (with Anirudh Sankar)}{We study the instrumental value of conceptual knowledge when making statistical decisions.
Such knowledge tells agents how unknown, payoff-relevant states relate.
It is distinct from the statistical knowledge gained from observing signals of those states.
We formalize this distinction in a tractable framework used by economists and statisticians.
Conceptual knowledge is valuable because it empowers agents to design more informative signals.
It is more valuable when states are more "reducible": when they can be explained with fewer common concepts.
Its value is non-monotone in the number of signals and vanishes when agents have infinitely many signals.
Agents who know more concepts can attain the same payoffs with fewer signals.
This is especially true when states are highly reducible.
}

\section{Working Papers}

\paper{\href{https://drive.google.com/file/d/11ZYBbbUYx-K0eJmqU4_jaIJIe7F8DdM0/view}{\bfseries How mechanistic explanations reshape learning and behavior: Evidence from a fertilizer choice experiment in Eastern Uganda} (with Anirudh Sankar, Robert Dulin, Vesall Nourani, Jess Rudder, Abraham Salomon, and Godfrey Taulya)}{Mechanistic explanations---descriptions of a system through the causal interactions of its parts---play a key role in human cognition and scientific progress.
Despite their importance, we lack systematic evidence on whether and how mechanistic explanations help lay decision-makers interpret information in complex economic environments.
We evaluate the causal impact of including mechanistic explanations in an information intervention: public demonstrations of fertilizer use for smallholder tomato farmers in Eastern Uganda.
In all demonstrations, extension officers showcased the impact of a recommended fertilizer recipe.
In the treatment group, officers also explained the mechanisms underlying the recipe’s effects---introducing the language of macronutrients and the causal processes linking nutrients, soil features, and plant growth.
We collect detailed data on beliefs and behaviors from 797 farmers in a lab-in-the-field experiment.
Treated farmers are better able to generalize from mechanisms to update beliefs about the returns to fertilizers, substitute and arbitrage among fertilizers based on nutrient content, and exhibit better understanding of the principles of nutrient and soil science.
In an incentivized fertilizer application task, they achieved 9\% higher simulated profits by selecting more agronomically sound fertilizer recipes, without increasing costs.
}
\vskip1em
\paper{\href{https://arxiv.org/abs/2401.03607}{\bfseries Learning about a changing state}}{A long-lived Bayesian agent observes costly signals of a time-varying state.
He chooses the signals' precisions sequentially, balancing their costs and marginal informativeness.
I compare the optimal myopic and forward-looking precisions when the state follows a Brownian motion.
I also compare the myopic precisions induced by other Gaussian processes.
}

\section{Peer-Reviewed Publications}

\paper{\href{https://doi.org/10.1016/j.econlet.2022.110640}{\bfseries Gender sorting among economists: Evidence from the NBER} \\ \emph{Economics Letters}, 2022}{I compare the co-authorship patterns of male and female economists, using historical data on National Bureau of Economic Research working papers.
Men tended to work in smaller teams than women, but co-authored more papers and so had more co-authors overall.
Both men and women had more same-gender co-authors than we would expect if co-authorships were random.
This was especially true for men in Macro/Finance.
}
\vskip1em
\paper{\href{https://doi.org/10.1080/00779954.2020.1806340}{\bfseries COVID-19, lockdown and two-sided uncertainty} (with Arthur Grimes) \\ \emph{New Zealand Economic Papers}, 2022}{When COVID-19 struck, the government had two choices: enter lockdown immediately or delay its decision.
Delaying would have allowed more information to emerge about health and economic impacts, and preserved the option to act later.
However, delaying may also have destroyed the option to eradicate COVID-19.
We model the government’s decisions under the health and economic uncertainty generated by COVID-19.
Our model captures both two-sided uncertainty and the dynamic consequences that flow from the government’s initial decision.
}
\vskip1em
\paper{\href{https://doi.org/10.1016/j.respol.2021.104421}{\bfseries Research funding and collaboration} (with Jason Gush, Shaun C. Hendy, and Adam B. Jaffe) \\ \emph{Research Policy}, 2022}{We analyze whether research funding contests promote co-authorship.
Our analysis combines Scopus publication records with data on the Marsden Fund, the premier source of funding for basic research in New Zealand.
We use fixed-effect models to analyze within-researcher-pair variation in co-authorship.
Among pairs who ever co-authored or co-proposed, co-authorship was 13.8 percentage points more likely in a given year if they had co-proposed during the previous ten years than if they had not.
This co-authorship rate was not significantly higher among funded pairs.
However, when we increase post-proposal publication lags towards the length of a typical award, we find that funding, rather than participation, promotes co-authorship.
}
\vskip1em
\paper{\href{https://doi.org/10.1080/00343404.2020.1802418}{\bfseries Relatedness, complexity and local growth} (with David C. Maré) \\ \emph{Regional Studies}, 2021}{We derive a measure of the relatedness between economic activities based on weighted correlations of local employment shares.
Our approach recognizes variation in the extent of local specialization and adjusts for differences in data quality between cities.
We use our measure to estimate activity and city complexity, and examine the contribution of relatedness and complexity to urban employment growth in New Zealand.
Relatedness and complexity are complementary in promoting employment growth in New Zealand’s largest cities, but do not contribute to employment growth in its smaller cities.
}
\section{Technical Notes}

\paper{\href{https://arxiv.org/abs/2404.00784}{\bfseries Estimating sample paths of Gauss-Markov processes from noisy data}}{I derive the pointwise conditional means and variances of an arbitrary Gauss-Markov process, given noisy observations of points on a sample path.
These moments depend on the process's mean and covariance functions, and on the conditional moments of the sampled points.
I study the Brownian motion and bridge as special cases.
}
\vskip1em
\paper{\href{https://www.iza.org/publications/dp/13642}{\bfseries Delineating functional labour market areas with estimable classification stabilities} (with David C. Maré)}{We describe an unsupervised method for delineating functional labour market areas (LMAs) in national commuting networks.
Our method uses the Louvain algorithm, which we extend to support top-down hierarchical LMA classification and estimable classification stabilities.
We demonstrate our method using historical Census commuting data from New Zealand.
}
\section{Awards}

\datedentry{2020}{William Georgetti Scholarship (for ``best brains'' in New Zealand)}
\datedentry{2017}{Seamus Hogan Memorial Prize (for best thesis in economics at Canterbury)}
\datedentry{2017}{Cook Memorial Prize (for top honors mathematics student at Canterbury)}
\datedentry{2016}{Sir Frank Holmes Prize (for top undergraduate economics student in New Zealand)}

\section{Research and Professional Experience}

\subsection{Stanford University}
\datedentry{2021--22}{Research Assistant to Matthew Jackson}

\subsection{Motu Economic and Public Policy Research}
\datedentry{2018--20}{Research Analyst}

\subsection{PwC New Zealand}
\datedentry{2015--16}{Actuarial Consulting Intern}

\section{Teaching Experience}

\subsection{Stanford University}
\datedentry{2025--}{Teaching Assistant for ECON 178 (Behavioral Economics)}
\datedentry{2025}{Teaching Assistant for ECON 137 (Decision Modeling and Information)}
\datedentry{2025}{Teaching Assistant for BUSGEN 143 (Finance, Corporations, and Society)}
\datedentry{2023--24}{Teaching Assistant for ECON 43 (Personal Finance)}
\datedentry{2022--24}{Teaching Assistant for ECON 135 (Foundations of Finance)}

\subsection{University of Canterbury}
\datedentry{2017}{Teaching Assistant for FINC 331 (Financial Economics)}
\datedentry{2016--17}{Teaching Assistant for ECON 208 (Intermediate Microeconomics)}
\datedentry{2016--17}{Teaching Assistant for ECON 321 (Microeconomic Theory)}

\section{Professional Service}
\entry{Referee for \emph{Research Policy}}

\section{Conference Presentations}

\datedentry{2023}{Society for Judgment and Decision Making Annual Conference (Poster), San Francisco}
\datedentry{2021}{NBER Summer Institute (Science of Science Funding)}
\datedentry{2021}{American Economic Association Annual Meeting}
\datedentry{2019}{New Zealand Association of Economists Conference, Wellington}
\datedentry{2018}{New Zealand Association of Economists Conference, Auckland}
\datedentry{2017}{European Group of Risk and Insurance Economists Annual Seminar, London}

\end{document}
